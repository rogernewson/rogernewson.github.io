% This is a LaTeX input file.
%
% A '%' character causes TeX to ignore all remaining text on the line,
% and is used for comments like this one.

\documentclass{article}      % Specifies the document class

                             % The preamble begins here.
\title{Identity of Somers' $D$ and the rank biserial correlation coefficient}  % Declares the document's title.
\author{Roger Newson}      % Declares the author's name.
\date{21 February, 2008}      % Deleting this command produces today's date.

\newcommand{\ip}[2]{(#1, #2)}
                             % Defines \ip{arg1}{arg2} to mean
                             % (arg1, arg2).

%\newcommand{\ip}[2]{\langle #1 | #2\rangle}
                             % This is an alternative definition of
                             % \ip that is commented out.

%
% Set margins
% (which are explained in Figure C3 of LaTeX User's Guide
% and which CAN BE RESET BY EDITORS AT ANY TIME AS FAR AS I CARE!!!!!!!!
% - RBN.)
%
\setlength{\topmargin}{-0.5in}
%\setlength{\headsep}{0.25in}
\setlength{\oddsidemargin}{0in}
\setlength{\evensidemargin}{0in}
\setlength{\textwidth}{6.5in}
\setlength{\textheight}{10in}

%
% Set page style (headers and footers)
%
\pagestyle{myheadings}
\markboth{\textit{Somers' $D$ and the rank biserial correlation coefficient}}
{\textit{Identity of Somers' $D$ and the rank biserial correlation coefficient}}

\begin{document}             % End of preamble and beginning of text.

\maketitle                   % Produces the title.

\section{Formulas}

We assume that there is a single sample of $N$ sampling units, partitioned into 2 subsamples (Subsample 1 and Subsample~2),
of $N_1$ and $N_2$ units, respectively,
such that $N_1+N_2=N$. For $h\in \{1,2\}$ and $1\le i\le N_h$,
denote by $Y_{hi}$ the outcome measure for the $i$th unit in Subsample~$h$,
and define $X_{hi}=h$, so that the ordinal $X$--variable indicates membership of the second subsample, rather than the first.

We note that, for each $h$ and $i$,
\begin{equation}
N = \#\{(j,k): Y_{jk}<Y_{hi}\} + \#\{(j,k): Y_{jk}=Y_{hi}\} + \#\{(j,k): Y_{jk}>Y_{hi}\},
\label{eq:eqseq8}
\end{equation}
where, for a set $S$, $\#S$ indicates the cardinality, or number of members, of $S$.
We can now define formally the rank of the $i$th member of Subsample~$h$ as
\def\half{{\frac{\scriptstyle 1}{\scriptstyle 2}}}
\begin{equation}
R_{hi} = \half + \half \#\{(j,k): Y_{jk}=Y_{hi}\} + \#\{(j,k): Y_{jk}<Y_{hi}\},
\label{eq:eqseq9}
\end{equation}
which implies that ranks can range from 1 to $N$,
and that units in a subset with tied $Y$--values are assigned the common mean rank that they would have had,
if they had been ordered randomly.
To simplify the algebra used with mean ranks, we may prefer to work with the linear transformation
\begin{equation}
Q_{hi} = 2R_{hi} - (N+1) = \#\{(j,k): Y_{jk}<Y_{hi}\} - \#\{(j,k): Y_{jk}>Y_{hi}\},
\label{eq:eqseq10}
\end{equation}
as implied by (\ref{eq:eqseq8}) and (\ref{eq:eqseq9}).
The sample mean rank, and mean transformed rank, for Subsample~$h$ are defined as
\begin{equation}
\bar R_{h} = N_h^{-1} \sum_{i=1}^{N_h} R_{hi}, \quad \bar Q_{h} = N_h^{-1} \sum_{i=1}^{N_h} Q_{hi} = 2 \bar R_{h}-(N+1).
\label{eq:eqseq11}
\end{equation}
Note that
\begin{eqnarray}
\bar Q_{h} \quad =&& N_h^{-1} \sum_{i=1}^{N_h} \# \{j: Y_{hj} < Y_{hi} \}
           \quad + \quad N_h^{-1} \sum_{i=1}^{N_h}  \# \{j: Y_{2-h+1,j}<Y_{hi} \} \nonumber \\
           &-& N_h^{-1} \sum_{i=1}^{N_h} \# \{j: Y_{hj} > Y_{hi} \}
           \quad - \quad N_h^{-1} \sum_{i=1}^{N_h}  \# \{j: Y_{2-h+1,j}>Y_{hi} \} \nonumber \\
=&& N_h^{-1} \sum_{i=1}^{N_h} \bigl[ \quad \# \{j: Y_{2-h+1,j}<Y_{hi} \} \quad - \quad \# \{j: Y_{2-h+1,j}>Y_{hi} \} \quad \bigr] ,
\label{eq:eqseq12}
\end{eqnarray}
because the terms involving within--sample ordinal contrasts of form $Y_{hi}<Y_{hj}$ and $Y_{hi}>Y_{hj}$ cancel out.

\def\RBC{{\rm RBC}}
We can now define the rank biserial correlation ($\RBC$) of Cureton (1956) as
\begin{equation}
\RBC = \frac{2}{N} \left( \bar R_2 - \bar R_1 \right) = N^{-1} \left( \bar Q_2 - \bar Q_1 \right).
\label{eq:eqseq13}
\end{equation}
Using (\ref{eq:eqseq12}), we see that
\def\sign{{\rm sign}}
\begin{eqnarray}
\label{eq:eqseq14}
\RBC \quad =&& \frac{N_1}{N} \frac{1}{N_1 N_2} \sum_{j=1}^{N_2} \quad \# \{k: Y_{1k}<Y_{2j} \}
     \quad - \quad \frac{N_1}{N} \frac{1}{N_1 N_2} \sum_{j=1}^{N_2} \quad \# \{k: Y_{1k}>Y_{2j} \} \nonumber \\
     &-& \frac{N_2}{N} \frac{1}{N_1 N_2} \sum_{k=1}^{N_1} \quad \# \{j: Y_{2j}<Y_{1k} \}
     \quad + \quad \frac{N_2}{N} \frac{1}{N_1 N_2} \sum_{k=1}^{N_1} \quad \# \{j: Y_{2j}>Y_{1k} \} \nonumber \\
     =&& \frac{1}{N_1 N_2} \sum_{j=1}^{N_2} \sum_{k=1}^{N_1} \sign(Y_{2j} - Y_{1k} )  \nonumber \\
     =&& \hat D(Y|X),
\end{eqnarray}
where $\sign(z)$ is 1 if $z>0$, $-1$ if $z<0$, and 0 if $z=0$,
and $\hat D(Y|X)$ is the sample estimate of Somers'~$D$ of $Y$ with respect to $X$,
based on the $Y_{hi}$ and the $X_{hi}$ (Somers, 1962).
The sample and population Somers'~$D$ parameters are discussed further in Newson (2006) and Newson (2002).

\section{References}

{\parindent=0pt

\smallskip
Cureton EE. Rank--biserial correlation.
\textsl{Psychometrika} 1956; \textbf{21}: 287--290.

\smallskip
Newson R. Confidence intervals for rank statistics: Somers'~$D$ and extensions.
\textsl{The Stata Journal} 2006; \textbf{6(3)}: 309--334.

\smallskip
Newson R. Parameters behind ``nonparametric'' statistics: Kendall's tau, Somers'~$D$ and median differences.
\textsl{The Stata Journal} 2002; \textbf{2(1)}: 45--64.

\smallskip
Somers RH. A new asymmetric measure of association for ordinal variables.
\textsl{American Sociological Review} 1962; \textbf{27}: 799-�811.

}


\end{document}               % End of document.
