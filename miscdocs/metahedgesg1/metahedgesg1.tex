% This is a LaTeX input file.
%
% A '%' character causes TeX to ignore all remaining text on the line,
% and is used for comments like this one.

\documentclass[a4paper]{article}      % Specifies the document class

                             % The preamble begins here.
\title{Formulas for estimating and pooling Hedges'~$g$ parameters in a meta-analysis}  % Declares the document's title.
\author{Roger B. Newson}      % Declares the author's name.
%\date{15 March, 2020}      % Deleting this command produces today's date.

\newcommand{\ip}[2]{(#1, #2)}
                             % Defines \ip{arg1}{arg2} to mean
                             % (arg1, arg2).

%\newcommand{\ip}[2]{\langle #1 | #2\rangle}
                             % This is an alternative definition of
                             % \ip that is commented out.

\usepackage{hyperref}

%
% Set margins
% (which are explained in Figure C3 of LaTeX User's Guide
% and which CAN BE RESET BY EDITORS AT ANY TIME AS FAR AS I CARE!!!!!!!!
% - RBN.)
%
\setlength{\topmargin}{-0.5in}
%\setlength{\headsep}{0.25in}
\setlength{\oddsidemargin}{0in}
\setlength{\evensidemargin}{0in}
\setlength{\textwidth}{6.5in}
\setlength{\textheight}{10in}

%
% Set page style (headers and footers)
%
\pagestyle{myheadings}
\markboth{\textit{Estimating and pooling Hedges'~$g$ in a meta--analysis}}
{\textit{Estimating and pooling Hedges'~$g$ in a meta--analysis}}

\begin{document}             % End of preamble and beginning of text.

\maketitle                   % Produces the title.

\section{Introduction}

The parameter Hedges'~$g$ is also known as the standardized mean difference (SMD).
It was advocated by Hedges (1981)\cite{hedges1981} as a measure of effect of a treatment,
to be derived from a 2-sample comparison between treated and untreated subjects,
which can be compared, and pooled in a meta-analysis, between studies where different outcomes are measured for the same treatment comparison.
It is expected to be useful in areas such as physiotherapy,
where there may be no consensus as to how to measure the efficacy of a treatment,
and where different symptom scores are used in different studies.
An example of an application of Hedges'~$g$ appears in Diong \textit{et al.}, 2016\cite{diongetal2016}.
Having done a meta--analysis of Hedges'~$g$ parameters on a messy set of journal articles,
the current author would like to share some of the formulas used with other meta--analysts for future reference.

\section{Formulas}

Hedges'~$g$ was defined by Hedges (1981)\cite{hedges1981} as a general measure of treatment effect,
valid when comparing 2 treatment groups regarding a Normally--distributed outcome
that is equally variable in the 2 sub--populations from which the treatment groups are sampled.
Following Hedges, we assume that there are $K$ studies (or study-outcomes),
each featuring 2 treatment groups, an intervention group (with index 1) and a control group (with index 0).
For each $i$ from 1 to $K$, we assume that there are $n_{0i}$ subjects in the control group,
and $n_{1i}$ subjects in the intervention group,
and that the outcomes measured in Study~$i$ are $Y_{0ij}$ for $j$ from 1 to $n_{0i}$ in the control group
and $Y_{1ij}$ for $j$ from 1 to $n_{1i}$ in the intervention group.
We assume that, for each $i$, the control values $Y_{0ij}$ are sampled from a common Normal distribution
with mean $\mu_{0i}$ and standard deviation (SD) $\sigma_{0i}$,
and that the intervention values $Y_{1ij}$ are sampled from another common Normal distribution
with mean $\mu_{1i}$ and SD $\sigma_{1i}$.

The standard sampling estimators for $\mu_{0i}$ and $\mu_{1i}$ are the respective sample means,
\def\Ybar{{\overline{Y}}}
\begin{equation}
\begin{array}{rl}
\Ybar_{0i} = & n_{0i}^{-1}\sum_{j=1}^{n_{0i}} Y_{0ij}, \\
\Ybar_{1i} = & n_{1i}^{-1}\sum_{j=1}^{n_{1i}} Y_{1ij},
\label{eq:eqseq1}
\end{array}
\end{equation}
and the standard sample estimators for $\sigma_{0i}$ and $\sigma_{1i}$ are the respective sample SDs,
\def\SDbar{{\overline{SD}}}
\begin{equation}
\begin{array}{rl}
\SDbar_{0i} = & \sqrt{ \left( n_{0i} - 1 \right)^{-1}\sum_{j=1}^{n_{0i}} \left( Y_{0ij} - \Ybar_{0i} \right)^2 }, \\
\SDbar_{1i} = & \sqrt{ \left( n_{1i} - 1 \right)^{-1}\sum_{j=1}^{n_{1i}} \left( Y_{1ij} - \Ybar_{1i} \right)^2 }. \\
\label{eq:eqseq2}
\end{array}
\end{equation}
However, if we think that we can assume that the control and intervention SDs are equal in each study ($\sigma_{0i}=\sigma_{1i}=\sigma_i$),
then we can estimate each common study--specific SD $\sigma_i$ by weighting the two squared treatment--specific SDs to give the estimate
\begin{equation}
\SDbar_i = \sqrt{ {{(n_{0i}-1)\SDbar^2_{0i} + (n_{1i}-1)\SDbar^2_{1i}}\over{n_{0i}+n_{1i}-2}} }.
\label{eq:eqseq3}
\end{equation}
This estimate enables us to estimate the difference between the intervention and control study mean in each study, expressed in units of the common SD.
In the $i$th study, this difference, in the study population at large, is the $i$th population Hedges'~$g$, defined as
\begin{equation}
\gamma_i = \left(\mu_{1i}-\mu_{0i}\right)/\sigma_i,
\label{eq:eqseq4}
\end{equation}
and is estimated using the $i$th sample Hedges'~$g$, defined as the estimator
\begin{equation}
g_i = \left( \Ybar_{1i} - \Ybar_{0i} \right) / \SDbar_i .
\label{eq:eqseq5}
\end{equation}
For the purposes of making funnel plots or carrying out heterogeneity tests,
we may use the standard--error formula for the equal--variance $t$-test to define an approximate standard error for the individual $g_i$
using the variance formula of Hedges (1981)\cite{hedges1981}.
If we assume equal variances in the intervention and control groups,
then each individual $g_i$ has the standard error
\begin{equation}
SE\left(g_i\right) = \sqrt{ {1\over n_{0i}} + {1 \over n_{1i}} } ,
\label{eq:eqseq20}
\end{equation}
using the standard equal--variance linear--regression theory of Seber (1977)\cite{seber1977}.
This standard error can be used with $g_i$ to generate a confidence interval and a $P$--value for $\gamma_i$,
using the ``Student'' $t$--distribution with $n_{0i}+n_{1i}-2$ degrees of freedom.

In a meta-analysis, we wish to estimate the weighted mean of the $\gamma_i$, which is equal to the common value if all the $\gamma_i$ are equal.
Usually, this meta--population parameter is defined as a weighted sum of the $\gamma_i$,
using weights $w_i$ that we hope are approximately inversely proportional to the sampling variance of the $g_i$.
(These weights are frequently the pooled intervention and control sample numbers $n_i = n_{1i}+n_{0i}$.)
It is defined as
\def\gammabar{{\overline{\gamma}}}
\begin{equation}
\gammabar = {{\sum_{i=1}^K w_i \gamma_i}\over{\sum_{i=1}^K w_i}}.
\label{eq:eqseq6}
\end{equation}
If these weights $w_i$ can be estimated using consistent estimators $W_i$,
then the estimator for the meta--population $\gammabar$ is the weighted sum
\def\gbar{{\overline {g}}}
\begin{equation}
\gbar = {{\sum_{i=1}^K W_i \, g_i}\over{\sum_{i=1}^K W_i}}.
\label{eq:eqseq7}
\end{equation}
In the case where the $w_i$ are the pooled $n_i$, the $W_i$ are also the $n_i$.
However, the $w_i$ might be inverse sampling variances for the $g_i$,
and then the $W_i$ may be inverse variance estimates.

\subsection{The connection with Somers'~$D$}

A possible justification for Hedges'~$g$ is as a less robust, but more widely available, substitute for Somers'~$D$ (Newson, 2006)\cite{newson2006}.
Somers'~$D$ can be viewed as a common currency for associations,
which can be compared, and meta--analysed, between associations between variables defined on a variety of scales (Newson, 2015)\cite{newson2015}.
In the two--sample case considered here,
it measures how little overlap there is between the 2 treatment groups (intervention and control).

If the assumption of equal variances between treatment groupss in the same study for the same outcome is true,
then the $i$th Hedges'~$g$ is related to the corresponding Somers'~$D$
using the formula
\begin{equation}
D_i = 2 \Phi\left( \gamma_i\over\sqrt{2}\right) - 1,
\label{eq:eqseq8}
\end{equation}
where $\Phi(\cdot)$ is the cumulative standard Normal distribution function,
and $D_i$ is the Somers'~$D$, in the $i$th study sub--population,
of outcome with respect to treatment (defined as 0 for control, 1 for intervention).

In practice, of course, we cannot always estimate Somers'~$D$ for each study in a meta-analysis,
except if we have the original data for each study.
Hedges'~$g$ has the advantage that journal articles usually provide enough information
for the user to estimate the study--treatment means $\mu_{0i}$ and $\mu_{1i}$
and the study--treatment SDs $\sigma_{0i}$ and $\sigma_{1i}$,
allowing us to estimate Hedges'~$g$ for each study, using the formula (\ref{eq:eqseq5}).

\subsection{Alternative estimates for the study parameters}

Usually, when we do a meta-analysis, the report from a component study provides estimates for the study--specific $\mu_{0i}$, $\mu_{1i}$,
$\sigma_{0i}$ and $\sigma_{1i}$.
In most cases, these are the corresponding $\Ybar_{0i}$, $\Ybar_{1i}$, $\SDbar_{0i}$ and $\SDbar_{1i}$.
However, sometimes these are not available. In these cases, the meta--analyst must find substitutes.

Sometimes, instead of treatment--group means and SDs,
the study reports give treatment--group means with confidence limits.
For the $i$th study, and for $h$ equal to zero for the control sub--sample and 1 for the intervention sub--sample,
the lower and upper confidence limits for $\Ybar_{hi}$ may be denoted
\def\Ybarlowerhi{{\Ybar_{hi}^{\rm (lower)}}}
\def\Ybarupperhi{{\Ybar_{hi}^{\rm (upper)}}}
$\Ybarlowerhi$ and $\Ybarupperhi$, respectively. The estimated SD is then given by the alternative formula
\def\invt{{\rm invt}}
\def\half{{1\over 2}}
\begin{equation}
\SDbar_{hi} = {{\sqrt{n_{hi}} \left( \Ybarupperhi-\Ybarlowerhi \right)}\over{\invt \left( n_{hi}-1 , 1-\half\alpha \right) - \invt \left( n_{hi}-1 , \half\alpha \right) }}.
\label{eq:eqseq9}
\end{equation}
Here,
$\alpha$ is defined so that the confidence interval has confidence level $100(1-\alpha)$,
so that $\alpha=0.05$ for a 95~percent confidence interval,
and $\invt(a,b)$ is defined as the inverse cumulative Student's $t$--distribution function, with $a$ degrees of freedom, of $b$,
as implemented in Stata using the Stata statistical function \texttt{invt()} (StataCorp, 2015)\citet{statacorp}.

Alternatively, a report from a component study may provide confidence limits only for the difference between intervention and control study means,
and not for the study means themselves.
In these cases, we may denote the lower and upper confidence limits for the $i$th treated--control difference as
\def\diffloweri{{{\rm diff}_{i}^{\rm (lower)}}}
\def\diffupperi{{{\rm diff}_{i}^{\rm (upper)}}}
$\diffloweri$ and $\diffupperi$, respectively.
We may then reconstruct the standard error of the difference from the confidence limits,
using the formula
\def\SEdiffi{{{\rm SE}_i^{\rm diff}}}
\begin{equation}
\SEdiffi = {{\diffupperi - \diffloweri}\over{\invt \left( n_{0i}+n_{1i}-2 , 1-\half\alpha \right) - \invt \left( n_{0i}+n_{1i}-2 , \half\alpha \right) }},
\label{eq:eqseq10}
\end{equation}
where $\alpha$ and $\invt(\cdot,\cdot)$ are defined as before.
Assuming that the population SDs are equal in the control and treated subpopulations for the $i$th study,
we may then estimate the common SD $\sigma_i$ using the formula
\begin{equation}
\SDbar_i = { { \SEdiffi }\over{ \sqrt{ 1/n_{0i} + 1/n_{1i} } } }.
\label{eq:eqseq11}
\end{equation}
We can then substitute (\ref{eq:eqseq11}) into (\ref{eq:eqseq5}) to estimate the $i$th Hedges'~$g$.

Alternatively, the study may present no means, standard deviations or standard errors, but only medians and other percentiles,
or even only a median and an interpercentile range.
In this case, the investigators may have done this because the outcome does not have a perfect Normal distribution.
However, the best possible approximation to a Hedges'~$g$ may still be to treat the median as the mean,
and to estimate the SD from the percentiles, or from the interpercentile range.
\def\xibar{{\overline{\xi}}}
For each probability $q$, we will denote by $\xi_{hi}^{(q)}$ the $100q$th population percentile, in treatment group~$h$ of study~$i$,
and we will denote by $\xibar_{hi}^{(q)}$ the corresponding estimate of $\xi_{hi}^{(q)}$ from the study report.
The treatment--group mean is then estimated as the corresponding median,
\begin{equation}
\Ybar_{hi} = \xibar_{hi}^{(0.5)}.
\label{eq:eqseq12}
\end{equation}
To estimate the SD, we will need 2 percentiles, or at least their difference.
For instance, if the percentiles given are the median, the 25th percentile, and the 75th percentile,
then we usually estimate the SD using the 25th and 75th percentiles,
or even using just their difference, known as the interquartile range,
which is sometimes given in a study report without the original percentiles.
\def\qlower{{q^{({\rm lower})}}}
\def\qupper{{q^{({\rm upper})}}}
We will denote by $\qlower$ and $\qupper$ the lower and upper proportions corresponding to the reported percentiles,
such that the study report has presented percentiles $100\qlower$ and $100\qupper$,
or just their difference.
We will denote by $\xi_{hi}^{({\rm lower})}$ and $\xi_{hi}^{({\rm upper})}$
the lower and upper population percentiles for treatment group~$h$ in study~$i$,
and denote by  $\xibar_{hi}^{({\rm lower})}$ and $\xibar_{hi}^{({\rm upper})}$
the corresponding sample percentiles, which were given in the report
(or at least their difference was).
The population SD for treatment group~$h$ of study~$i$ is then given by the formula
\begin{equation}
\sigma_{hi} = {{ \xi_{hi}^{({\rm upper})} - \xi_{hi}^{({\rm lower})} }\over{ \Phi^{-1}\left(\qupper\right) - \Phi^{-1}\left(\qlower\right) }},
\label{eq:eqseq13}
\end{equation}
where $\Phi^{-1}(\cdot )$ is the inverse standard Normal cumulative distribution function,
computed in Stata using the \texttt{invnormal()} function\cite{statacorp}.
This SD can therefore be estimated using the formula
\begin{equation}
\SDbar_{hi} = {{ \xibar_{hi}^{({\rm upper})} - \xibar_{hi}^{({\rm lower})} }\over{ \Phi^{-1}\left(\qupper\right) - \Phi^{-1}\left(\qlower\right) }}.
\label{eq:eqseq14}
\end{equation}
Note that this formula depends only on the difference between the upper and lower percentiles.
Therefore, if $\qlower=0.25$ and $\qupper=0.75$,
then the formula~(\ref{eq:eqseq14}) requires only the interquartile range.

\subsection{The positive--beneficial Hedges'~$g$}

Sometimes, the meta--analysis is complicated by the fact that some of the study outcomes are higher if the subject is doing well,
and other study outcomes are higher if the subject is doing badly.
In this case, we need to summarize the results using a revised Hedges'~$g$,
known as a positive--beneficial Hedges'~$g$,
which will be positive if subjects do better (on average) in the intervention group
(as is expected to happen if the treatment is beneficial),
and negative if subjects do less well (on average) in the intervention group
(as is expected to happen if the treatment is harmful).

To deal with this possibility, we define the study--specific beneficiality sign for the $i$th study (or study--outcome) as
\begin{equation}
\beta_i = \left\{
\begin{array}{ll}
1,  & \mbox{if higher $Y_{hij}$ values are  better,} \\
-1, & \mbox{if lower $Y_{hij}$ values are better.}
\end{array}
\right.
\label{eq:eqseq15}
\end{equation}
We can then redefine the $i$th study--specific population Hedges'~$g$ by modifying formula (\ref{eq:eqseq4}) as
\begin{equation}
\gamma_i^* = \beta_i \gamma_i,
\label{eq:eqseq16}
\end{equation}
and redefine the corresponding $i$th study--specific sample Hedges'~$g$ by modifying (\ref{eq:eqseq5}) as
\begin{equation}
g_i^* = \beta_i g_i,
\label{eq:eqseq17}
\end{equation}
for which an approximate standard error, for funnel plots and heterogeneity ttests, is once again given by the variance formula of Hedges (1981)\cite{hedges1981}.

The parameter we need to estimate, in order to summarize the benefit of intervention, is then
\begin{equation}
\gammabar^* = {{\sum_{i=1}^K w_i \gamma_i^*}\over{\sum_{i=1}^K w_i}},
\label{eq:eqseq18}
\end{equation}
which we estimate using the weighted mean
\begin{equation}
\gbar^* = {{\sum_{i=1}^K W_i \, g_i^*}\over{\sum_{i=1}^K W_i}}.
\label{eq:eqseq19}
\end{equation}
Note that, if this method is to work as advertized,
then the $\beta_i$ should be decided \textit{a priori},
before knowing the directions of the $g_i$.

\section{Further extensions}

\subsection{Clustered and weighted Huber variances}

In the simplest case, all the $K$ studies can be assumed mutually independent.
However, this is not necessarily the case.
In some complicated meta--analyses,
we have multiple outcomes per study,
each with its own beneficiality sign,
and maybe also with its own weight,
because some outcomes were measured in more subjects than others.

In those cases, we can make the observational unit a study--outcome instead of a study,
and use clustered Huber variances to estimate the standard errors and confidence limits
for estimating the parameter (\ref{eq:eqseq18}) using the statistic (\ref{eq:eqseq19}).
These clustered and weighted Huber variances use the assumption that we are sampling studies independently from a population of studies,
instead of sampling study--outcomes from a population of study--outcomes.
This method is essentially a special case of the more general meta--regression methods
recommended by Hedges~\textit{et al.}, 2010\cite{hedges2010}.

\subsection{Binary outcomes}

Sometimes, the same meta--analysis can combine study--outcomes with quantitative outcomes and study--outcomes with binary outcomes.
One possible way of combining these might be to convert the Hedges'~$g$ estimates to a Somers'~$D$ scale using \ref{eq:eqseq8},
and to represent the binary--outcome comparisons as differences between proportions,
possibly signing both types of comparisons to be positive--beneficial.
This possibility is justified because a difference between proportions is a special case of Somers'~$D$\cite{newson2015}.
\textit{However}, we may be presenting our meta--analysis to an audience who may be unfamiliar with Somers~$D$
and who may think that they understand odds ratios better.
And, sometimes, the studies in the meta--analysis may not even report the proportions of positive results,
or even the corresponding odds,
but only an odds ratio between the two corresponding odds.

For such eventualities, Chinn (2000)\cite{chinn2000} developed a method for converting an odds ratio to a Hedges'~$g$,
assuming each binary outcome in each treatment group to be derived from a Normally--distributed latent variable
by being assigned the value of 1 if and only if the latent variable exceeds a threshold $\theta$.
This Normally--distributed variable is known as the Normal-equivalent deviate (NED).
Without loss of generality, we can assume that the NED has a standard deviation of 1,
because otherwise a standard--Normal--equivalent deviate (SNED) can be derived by dividing the NED by its standard deviation.
And,
if the intervention and control groups have the same standard deviation for the NED
and differ only in the value of the threshold SNED above which the binary outcome is 1,
then the Hedges'~$g$ that we might want to estimate is simply the difference between the control SNED threshold $\theta_0$
and the intervention SNED threshold $\theta_1$
(if a positive binary outcome is thought to be beneficial),
or the difference $\theta_1-\theta_0$ between the intervention and control SNED thresholds
(if a positive binary outcome is thought to be harmful).

The Chinn method is based on using a second latent variable, the standard logistic--equivalent deviate (SLED),
as a proxy for the SNED.
The SLED is assumed to have a standard--logistic distribution,
which is defined on the real line, and has cumulative distribution function
\begin{equation}
\Lambda(x) = {1 \over {1-\exp\left(x\right)}} \, ,
\label{eq:eqseq21}
\end{equation}
and has mean $0$ and variance $\pi^2/3$.
In a logistic regression model, we can fantasize that the binary outcome is positive if and only if the SLED exceeds a standard--logistic threshold.
And, if there are two different standard--logistic thresholds for the intervention and control groups,
then the difference between the control standard--logistic threshold and the intervention standard--logistic threshold
is equal to the log of the intervention/control odds ratio for a positive binary outcome.
The Chinn method assumes that the SNED is monotonically increasing in the SLED,
and is given by the formula
\def\SNED{{\rm SNED}}
\def\SLED{{\rm SLED}}
\def\odds{{\rm Odds}}
\begin{equation}
\SNED = \Phi^{-1}\left[ \Lambda \left( \SLED \right) \right] \approx {\pi \over \sqrt{3}}\SLED \, ,
\label{eq:eqseq22}
\end{equation}
which Chinn argued was a good linear approximation,
as the standard Normal and standard logistic distributions are both bell--shaped and symmetrical around a common mode of 0.
It follows that the difference between control and intervention standard--Normal thresholds (also known as the Hedges'~$g$)
can be approximated as
\def\OR{{\rm OR}}
\begin{equation}
\gamma \approx {\pi \over \sqrt{3}} \ln \OR \, ,
\label{eq:eqseq23}
\end{equation}
where $\OR$ is the intervention/control odds ratio.
So, in a meta--analysis, if the $i$th outcome is binary,
then the $i$th control--intervention Hedges'~$g$ estimate can be derived approximately from the $i$th intervention/control odds ratio
using the approximation
\begin{equation}
g_i \approx {\pi \over \sqrt{3}} \ln \widehat\OR_i \, ,
\label{eq:eqseq24}
\end{equation}
where $\widehat\OR_i$ is the estimated intervention/control odds ratio for the $i$th outcome.
Note that we can reverse the sign of the Hedges $g$ to be positive--beneficial,
if we think that a positive binary outcome is a good thing instead of a bad thing,
in which case we would prefer the intervention--control threshold difference to be negative,
both on the SLED scale and on the SNED scale.
And, if we think that we have a reliable standard error for $\ln \widehat \OR_i$,
then we can use that in the meta-analysis.
Otherwise, we can use clustered and sample--size--weighted Huber variances.

If we are fortunate enough to have intervention and control odds for each study,
instead of only having the intervention/control or control/intervention odds ratio,
then we can estimate the SNED threshold ($\theta_{ji}$ for Treatment~$j$ and Outcome~$i$)
directly from the corresponding SLED threshold (given by the log--odds of a positive binary outcome),
using (\ref{eq:eqseq22}).
Or, even better, we can estimate the SNED threshold more directly, using the formula
\begin{equation}
\theta_{ij} = \Phi^{-1}\left[ \odds_{ij}/\left( 1+\odds_{ij} \right) \right],
\label{eq:eqseq25}
\end{equation}
where $\odds_{ij}$ is the odds of a positive outcome
under Treatment~$i$ in the $j$th study--outcome.
We can then use the exact formula, instead of the Chinn approximate formula.
The intervention--control positive--beneficial Hedges'~$g$
is then $\theta_{1i}-\theta_{0i}$ (if a positive binary outcome is bad),
or $\theta_{0i}-\theta_{1i}$ (if a positive binary outcome is good).
Whitehead \textit{et al}. (1999)\cite{whitehead1999} recommended essentially this method,
phrasing it in terms of a lognormal latent variable,
and a standard--lognormal threshold equal to the odds,
which is the exponential of the standard--Normal threshold.

\begin{thebibliography}{10}

\bibitem{chinn2000}
Chinn S.
A simple method for converting an odds ratio to effect size for use in meta--analysis.
\textsl{Statistics in Medicine} 2000; \textbf{19}: 3127--3131.

\bibitem{hedges1981}
Hedges LV.
Distribution theory for Glass's estimator of effect size and related estimators.
\textsl{Journal of Educational Statistics} 1981; \textbf{6(2)}: 107--128.

\bibitem{hedges2010}
Hedges LV, Tipton E, Johnson MC.
Robust variance estimation in meta--regression with dependent effect size estimates.
\textsl{Research Synthesis Methods} 2010; \textbf{1}: 39--65.

\bibitem{diongetal2016}
Diong J, Allen N, Sherrington C.
Structured exercise improves mobility after hip fracture: a meta--analysis with meta--regression.
\textsl{British Journal of Sports Medicine} 2016; \textbf{50}: 346--355.

\bibitem{newson2006}
Newson R.
Confidence intervals for rank statistics: Somers'~$D$ and extensions.
\textsl{The Stata Journal} 2006; \textbf{6(3)}: 309--334.
Download from \hfill \break
\mbox{\href{http://www.stata-journal.com/article.html?article=snp15_6}{\textsl{http://www.stata-journal.com/article.html?article=snp15\_6}}}

\bibitem{newson2015}
Newson RB.
Somers'~$D$: A common currency for associations.
Presented at the 21st UK Stata User Meeting, 10--11 September, 2015.
Download from \hfill \break
\mbox{\href{https://ideas.repec.org/p/boc/usug15/01.html}{\textsl{https://ideas.repec.org/p/boc/usug15/01.html}}}

\bibitem{seber1977}
Seber GAF.
\textsl{Linear regression analysis.}
New York, NY: John Wiley \& Sons; 1977.

\bibitem{statacorp}
StataCorp.
\textsl{Stata: Release 14. Statistical Software.}
College Station, TX: StataCorp LP; 2015.

\bibitem{whitehead1999}
Whitehead A, Bailey AJ, Elbourne D.
Combining summaries of binary outcomes with those of continuous outcomes in a meta--analysis.
\textsl{Journal of Biopharmaceutical Statistics} 1999; \textbf{9(1)}: 1�-16.

\end{thebibliography}

\end{document}               % End of document.
