% This is a LaTeX input file.
%
% A '%' character causes TeX to ignore all remaining text on the line,
% and is used for comments like this one.

\documentclass{article}      % Specifies the document class

                             % The preamble begins here.
\title{Confidence intervals for scenario means and their differences and ratios}  % Declares the document's title.
\author{Roger Newson}      % Declares the author's name.
\date{28 August, 2005}      % Deleting this command produces today's date.

\newcommand{\ip}[2]{(#1, #2)}
                             % Defines \ip{arg1}{arg2} to mean
                             % (arg1, arg2).

%\newcommand{\ip}[2]{\langle #1 | #2\rangle}
                             % This is an alternative definition of
                             % \ip that is commented out.

\begin{document}             % End of preamble and beginning of text.

\maketitle                   % Produces the title.

\section{Formulas}

Assume that we fit to a set of data a generalized linear model with $P$ parameters $\mathbf{\beta}=(\beta_1, \ldots, \beta_P)^T$
to a set of data with $N$ data points. We will denote the $Y$--value for the $i$th data point as $Y_i$,
we will denote the $j$th $X$--value for the $i$th data point as $X_{ij}$,
we will denote the conditional mean for the $i$th data point as
$\mu_i$, and we will denote its link function as $\eta_i$. The overall mean is defined as
\begin{equation}
M=N^{-1} \sum_{i=1}^N \mu_i \, ,
\label{eq:eqseq1}
\end{equation}
and its derivative with respect to the $j$th parameter is
\begin{equation}
G_j = { {\partial M}\over{\partial \beta_j} } = N^{-1} \sum_{i=1}^N { {\partial \mu_i} \over {\partial \eta_i} } X_{ij} \, ,
\label{eq:eqseq2}
\end{equation}
and the derivative of its log with respect to the $j$th parameter is
\begin{equation}
\Gamma_j = { \partial \over {\partial \beta_j} } \log(M) = \left( \sum_{i=1}^N \mu_i \right)^{-1} \sum_{i=1}^N { {\partial \mu_i} \over {\partial \eta_i} } X_{ij} \, .
\label{eq:eqseq3}
\end{equation}
Note that, except in the trivial case of a linear link function (such as the familiar identity link),
these gradients are \textit{not} the gradients arrived at by setting all the $X$-variates to their sample means.

To define confidence intervals for $M$ and $\log(M)$,
we define the $P$-column row vectors $\mathbf{G}$ and $\mathbf{\Gamma}$ by (\ref{eq:eqseq2}) and (\ref{eq:eqseq3}) respectively,
\def\cov{\mathrm{Cov}}
\def\var{\mathrm{Var}}
denote by $\cov(\mathbf{\beta})$ the covariance matrix of the vector parameter $\mathbf{\beta}$,
and we then have the estimates
\begin{equation}
\var\left( M \right) = \mathbf{G} \cov(\mathbf{\beta}) {\mathbf{G}}^T,
 \quad \var\left[\log(M)\right] = \mathbf{\Gamma} \cov(\mathbf{\beta}) {\mathbf{\Gamma}}^T \, ,
\label{eq:eqseq4}
\end{equation}
and calculate standard errors and symmetric confidence limits in the usual manner,
possibly exponentiating these confidence limits in the case of $\log(M)$ to derive asymmetric confidence limits for $M$.

To compare expected overall means under different scenarios, we usually want to estimate either their differences
or their ratios. Using out--of--sample prediction, we can fantasize that, under ``Scenario~$*$'', we have a sample of $N^*$
observations, and their hypothesized $X$--values are denoted $X^*_{ij}$ for the $j$th $X$--variate in the $i$th observation,
and their hypothesized expected $Y$--values and their link functions (assuming the same $\mathbf{\beta}$ as before) are denoted
$\mu^*_i$ and $\eta^*_i$, respectively, for the $i$th observation. The overall scenario mean is then
\begin{equation}
M^* = {N^*}^{-1} \sum_{i=1}^{N^*} \mu^*_i \, ,
\label{eq:eqseq5}
\end{equation}
and we can define vectors $\mathbf{G^*}$ and $\mathbf{\Gamma^*}$ analogously to (\ref{eq:eqseq2}) and (\ref{eq:eqseq3}), respectively,
and define confidence intervals for $M^*$ and its log using formulas similar to (\ref{eq:eqseq4}).
For a second scenario, denoted ``Scenario~$**$'', we might similarly assume a sample size of $N^{**}$,
define $X$--values $X^{**}_{ij}$, expected $Y$--values $\mu^{**}_i$,
link functions $\eta^{**}_i$, an overall scenario mean $M^{**}$,
and gradient vectors $\mathbf{G^{**}}$ and $\mathbf{\Gamma^{**}}$.
The difference $M^*-M^{**}$ between the expected overall means under the two scenarios has a variance estimated as
\begin{equation}
\var \left( M^*-M^{**} \right) \quad = \quad \left( \mathbf{G^*} - \mathbf{G^{**}} \right) \cov(\mathbf{\beta}) \left( \mathbf{G^*} - \mathbf{G^{**}} \right)^T \, ,
\label{eq:eqseq6}
\end{equation}
and the corresponding log ratio $\log(M^*/M^{**})$ has a variance estimated as
\begin{equation}
\var \left[\log\left( M^*/M^{**} \right)\right] \quad = \quad \left( \mathbf{\Gamma^*} - \mathbf{\Gamma^{**}} \right) \cov(\mathbf{\beta}) \left( \mathbf{\Gamma^*} - \mathbf{\Gamma^{**}} \right)^T \, .
\label{eq:eqseq7}
\end{equation}
We can therefore calculate standard errors and confidence limits for the scenario difference $M^* - M^{**}$,
and for the log scenario ratio $\log(M^*/M^{**})$, in the usual manner, and define asymmetric confidence limits for
$M^*/M^{**}$.

An important special case of the scenario ratio is the population unattributable fraction,
which is subtracted from one to define the population attributable fraction.
In the case of a cohort study, ``Scenario~*'' might represent a hypothetical version of our cohort
if they were all non--smokers and were the same in all other respects,
and ``Scenario~**'' might represent the cohort we actually have.
In the case of a case--control study, ``Scenario~**'' might represent the controls in our sample
(assumed to represent the population at large because of the rare-disease assumption),
and ``Scenario~*'' might represent a hypothetical sample who are all non--smokers, but
who are like the controls in our sample in all other respects. For further information
on these examples, see Bruzzi \textit{et al}. (1985) and Greenland and Drescher (1993).

\section{References}

{\parindent=0pt

\smallskip
Bruzzi P, Green SB, Byar DP, Brinton LA, Schairer C.
Estimating the population attributable risk for multiple risk factors using case--control data.
\textsl{American Journal of Epidemiology} 1985; \textbf{122(5)}: 904--914.

\smallskip
Greenland S, Drescher K. Maximum likelihood estimation of the attributable fraction from logistic models.
\textsl{Biometrics} 1993; \textbf{49}: 865--872.

}


\end{document}               % End of document.
