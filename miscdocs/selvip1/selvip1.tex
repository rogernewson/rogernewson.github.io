% This is a LaTeX input file.
%
% A '%' character causes TeX to ignore all remaining text on the line,
% and is used for comments like this one.

\documentclass[a4paper]{article}      % Specifies the document class

                             % The preamble begins here.
\title{Asymptotic standard errors for the logs of variance--inflated Poisson--variance means and of their ratios}  % Declares the document's title.
\author{Roger B. Newson}      % Declares the author's name.
%\date{29 January, 2016}      % Deleting this command produces today's date.

\newcommand{\ip}[2]{(#1, #2)}
                             % Defines \ip{arg1}{arg2} to mean
                             % (arg1, arg2).

%\newcommand{\ip}[2]{\langle #1 | #2\rangle}
                             % This is an alternative definition of
                             % \ip that is commented out.

\usepackage{graphics}
\usepackage{hyperref}

%
% Set margins
% (which are explained in Figure C3 of LaTeX User's Guide
% and which CAN BE RESET BY EDITORS AT ANY TIME AS FAR AS I CARE!!!!!!!!
% - RBN.)
%
\setlength{\topmargin}{-0.5in}
%\setlength{\headsep}{0.25in}
\setlength{\oddsidemargin}{0in}
\setlength{\evensidemargin}{0in}
\setlength{\textwidth}{6.5in}
\setlength{\textheight}{10in}

%
% Set page style (headers and footers)
%
\pagestyle{myheadings}
\markboth{\textit{Standard errors of logged variance--inflated Poisson--variance means}}
{\textit{Standard errors of logged variance--inflated Poisson--variance means}}

\begin{document}             % End of preamble and beginning of text.

\maketitle                   % Produces the title.

\section{Formulas}

\def\odds{{\rm odds}}
We consider Poisson--variance variables $Y_i$ (typically event counts),
with positive means $\mu_i$ and variance inflation (or deflation) factors $\phi_i$,
such that the variance of $Y_i$ is given by $\phi_i \mu_i$.
Models with such variables were among those considered by Wedderburn (1974)\cite{wedderburn1974}.
The variance inflation (or deflation) may be imagined to be caused by clustering and/or weighting and/or adjustment for covariates.
We focus on the case where there are two such variables $Y_0$ and $Y_1$,
having expectations $\mu_0$ and $\mu_1$, with sum $\mu_+=\mu_0+\mu_1$, proportions $p_i=\mu_i/\mu_+$,
and ratio $R=\mu_1/\mu_0=p_1/p_0=p_1/(1-p_1)=\odds(p_1)$.
Then we find, using Taylor polynomials,
that the variances of the logs of the $Y_i$ converge in ratio, as the $\mu_i$ become large, to
\def\Var{{\rm Var}}
\def\SE{{\rm SE}}
\begin{equation}
\Var\left[\ln(Y_i)\right] = \mu_i^{-2}\phi_i\mu_i = \phi_i/\mu_i.
\label{eq:eqseq1}
\end{equation}
If we assume an equal--dispersion model in which the variances of $Y_0$ and $Y_1$ have a common variance inflation factor $\phi$,
then the variance of their log ratio, assuming independence, converges in ratio to
\begin{equation}
\label{eq:eqseq2}
\begin{array}{rl}
\Var\left[ \ln(Y_1/Y_0) \right] = & \phi \left( {1 \over \mu_0} + {1 \over\mu_1} \right) \\
= & { \phi \over \mu_+ } \left( {1 \over { 1-p_1}} + { 1 \over p_1 } \right) \\
= & { \phi \over \mu_+ } \left( {{ R+1 } \over 1} + {{ R+1} \over R} \right) \\
= & { \phi \over \mu_+ } \left( {R^2 + 2R + 1 } \over R \right) \\
= & { \phi \over \mu_+ } { {\left( R+1 \right)^2} \over R } .
\end{array}
\end{equation}
It follows that, if $\ln(Y_1/Y_0)$ is the estimator for $\ln(R)$, then its asymptotic sampling standard error is given by
\begin{equation}
\SE\left[ \ln(R) \right] = \sqrt{ { \phi \over \mu_+ } { {\left( R+1 \right)^2} \over R } }.   
\label{eq:eqseq3}
\end{equation}
This formula can be useful in power calculations for $\ln( R )$,
if we think we have a good prior guess for the values of $\mu_+$ and $\phi$.
Note that the standard error formula still applies if we are testing null hypotheses other than $R=1$,
as will be the case when $Y_0$ and $Y_1$ are event counts,
associated with different exposure--time totals having a known exposure--time ratio.

The factor $(R+1)^2/R$ in (\ref{eq:eqseq3}) has the feature that
\begin{equation}
\label{eq:eqseq4}
\begin{array}{rl}
{d \over dR} \left[ { { (R+1)^2 } \over R } \right] = & \left[ R { d \over dR} (R+1)^2 - (R+1)^2 {dR \over dR} \right] / R^2 \\
= & \left[ 2(R+1)R - (R+1)^2 \right] / R^2 \\
= & \left( 2R^2 +2R -R^2 - 2R -1 \right) / R^2  \\
= & ( R^2-1 ) / R^2 \\
= & 1 - {1 \over R^2}.
\end{array}
\end{equation}
This expression is positive for $R>1$, negative for $R<1$, and zero for $R=1$.
So, for each value of $\phi/\mu_+$,
the standard error of $\ln(R)$ is a smile-shaped function of R, with a minimum at $R=1$.
The log transformation therefore does not stabilize variances perfectly.

\subsection{Applications to power calculations}

Power calculations for Poisson rate ratios fall into the general theory summarized in Newson (2004)\cite{newson2004}.
In that theory, the 5 quantities that may be calculated are power ($\gamma=1-\beta$ where $\beta$ is the probability of Type 2 error),
$\alpha$ (the probability of Type 1 error),
$\delta$ (the detectable difference between parameter values under 2 hypotheses),
$\sigma$ (the standard deviation of the influence function of the parameter being estimated),
and $n$ (the necessary number of sampling units).
Any one of these quantities can be calculated from the other 4 quantities.
However, for our purposes, we will assume that $n=1$, because the sum of several independent count variables is a count variable,
and, if the variables being summed are Poisson, then so is their sum.
So, we will assume that there is a single primary sampling unit,
composed of 2 variance--inflated Poisson variables,
with a common dispersion parameter $\phi$,
measuring overdispersion or underdispersion,
which might be caused by clustering and/or sampling--probability weighting,
or which might be caused by the component count variables themselves being overdispersed or underdispersed.

We assume that we are measuring the log ratio of the population means of the 2 count variables.
And we assume that, under the null hypothesis being tested, the ratio $R$ has the value of $R_0$.
This value may be 1, if we are testing a null hypothesis of equal mean counts.
Alternatively, $R_0$ may be a ratio between exposure times at risk,
if we are testing a null hypothesis of equal event rates per unit time at risk.
It is important to note that, if $R_0$ is an exposure--time ratio,
then the ratio $R$ between mean counts, under an alternative hypothesis,
will be equal to the product of 2 ratios,
namely the exposure--time ratio $R_0$
and the ratio of event rates per unit time at risk under that alternative hypothesis.
Therefore, if we are doing power calculations for the detection of a ratio of event rates per unit exposure,
then this rate ratio must be multiplied by $R_0$ to derive $R$ for input into a power calculation.
(Or, if the detectable ratio of mean counts $R$ is output from a power calculation,
then $R$ must be divided by $R_0$ in order to derive a detectable rate ratio per unit exposure.)

Under these assumptions, once we have defined $\mu_+$, $\phi$, $R$ and $R_0$,
we have $n=1$, $\delta=\ln(R/R_0)=\ln(R)-\ln(R_0)$,
and the standard deviation $\sigma$ of the influence function is given by (\ref{eq:eqseq3})
as a function of $R$, $\phi$ and $\mu_+$.
The quantities $n$, $\delta$ and $\sigma$ appear in the equations in Newson (2004)\cite{newson2004},
together with the values of $\gamma$ and $\alpha$ under the appropriate scenarios.
Given that $n$ is fixed at 1,
we can use these equations to derive any one of the 4 quantities $\delta$, $\sigma$, $\alpha$ and $\gamma$
from the other 3.
This can be done using the \texttt{powercal} package in Stata, described in Newson (2004)\cite{newson2004},
to do the power calculations for multiple scenarios,
in a dataset with 1 observation per scenario.
Each scenario will be defined by values of the 6 parameters $\gamma$, $\alpha$, $R$, $R_0$, $\mu_+$ and $\phi$.

\begin{thebibliography}{10}

\bibitem{newson2004}
Newson R.
Generalized power calculations for generalized linear models and more.
\textsl{The Stata Journal} 2004; \textbf{4(4)}: 379-401.

\bibitem{wedderburn1974}
Wedderburn RWM.
Quasi-likelihood functions, generalized linear models, and the Gauss�-Newton method.
\textsl{Biometrika} 1974; \textbf{61(3)}: 439�-447.

\end{thebibliography}

\end{document}               % End of document.
